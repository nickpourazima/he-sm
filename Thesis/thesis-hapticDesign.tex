\chapter{Haptic Design}
This chapter briefly touches on the field of haptics and delves into tactile research. This leads into a discussion of the overall design requirements, initial prototypes, and overall challenges overcome which led to the development of the final prototype, the vibrotactile array.

\section{Brief introduction to haptics}
\textbf{Haptics} are the field of research which concern the sense of touch as it applies to \textit{kinesthetic} and \textit{tactile} sensation. The tactile sense enables humans to perceive object properties through skin contact while the \textit{kinesthetic} or \textit{proprioceptive} sense lets one perceive the positions, movements, and forces on one's own body. 

The skin is lined with an array of sensory receptors which respond to mechanical pressure and distortions such as skin deformation. The \textit{lamellated} or \textit{pacinian corpuscles} (PC) are responsible for sensitivity to vibration and pressure. These rapidly adapting receptors are responsible for vibrotactile perception in glabrous skin. 

Sensitivity to a tactile stimulus grows with the area in contact with the skin and also improves with the stimulus duration until it reaches a point of saturation. When pressure is continuous an effect called \textit{haptic masking} (also known as the \textit{summation effect}) is possible. The overstimulation of the \textit{pacinian corpuscles} causes the brain to ignore these messages with a mechanical filtering system which lowers the perception threshold in order to focus on other important happenings. If this was not the case, a person could for example feel the pressure exerted by wearing clothing \cite{choi2013vibrotactile}. This phenomena is important to consider when dealing with haptic placement. As mentioned in \ref{tactileModality}, when the vibrotactiles were placed over a larger area span haptic masking was avoided and the results closely matched the auditory modality.

\subsection{Haptic Considerations}
The following questions arise based on extensive research done by Choi and Kuchenbecker \cite{choi2013vibrotactile} and are crucial concepts underpinning the creation a meaningful haptic.
\begin{enumerate}
    \item \emph{Can the user feel it?}
Perceptibility of vibrotactile stimuli is strongly dependent on the frequency of vibration. The minimum threshold is observed to be between 150-300Hz and can cover an area smaller than 0.1 micrometer. The absolute thresholds are dependent on factors such as body site, contact area, stimulus duration, stimulus waveform, contact force, skin temperature, presence of other masking stimuli, and age.

    \item \emph{Can the user distinguish between the different vibrotactile cues being displayed?}
This is quantified by the discrimination or \textit{difference threshold} also called the \textit{Just Noticeable Difference} (JND). It is defined as the smallest amount a stimulus intensity much change to produce a noticeable change in sensory experience. The JND is measured as a \textit{Weber fraction}:
${\Delta}$I/I = k or the ratio of difference threshold to the reference level.
Research into experimental psychology has deemed a 20-30\% difference in amplitude or frequency is necessary for robust discrimination between vibrotactile stimuli in practical applications.

    \item \emph{How strong does a certain vibrotactile cue feel to the user?}
\textit{Steven's power law} describes the relationship between the magnitude of a physical stimulus and its perceived intensity or strength. See Figure \ref{fig:StevensPowerLaw}
\begin{figure}[H]
    \includegraphics[width=\linewidth,height=\textheight,keepaspectratio]{Steven's_Power_Law}
    \label{fig:StevensPowerLaw}
\end{figure}
When a stimulus intensity \textit{I} is above its absolute threshold, humans perceives its magnitude as \begin{math}\Psi(I)\end{math}(perceptual strength). The exponent (dependent on stimulation freq) determines growth rate of the perceived magnitude, ranges from 0.35 to 0.86 for vibrotactiles. Perceived intensity is a function of freq and amplitude of vibration (also affecting perceived pitch).
    \item \emph{How good are users at judging timing of vibrotactile cues?}
Tactile perception is generally considered to have high temporal acuity.
Vibrotactile temporal resolution research cites a humans ability to distinguish successive pulses with a time gap as small as 5 ms (12000 BPM). This resolution is better than vision (25ms) but slightly worse than physiological experiments into the peripheral auditory system which cites a theoretical best case scenario of approximately 2 ms \cite{fishbach2001auditory} \cite{parsons2006neurobiology}
    \item \emph{Can Vibrotactile cues elicit any other perceptual effects?} Below 3 Hz is considered slow kinesthetic motion. Between 10-70Hz is the sensation of rough motion or fluttering and between 100-300Hz is the sensation of smooth vibration. Subjective quality of a vibrotactile stimulus can be controlled by modifying the envelope of the stimulus amplitude.
\end{enumerate}

\subsection{Vibrotactiles}
The exploration of touch actuation led to the evaluation of available vibrotactiles. The following is a thorough breakdown to inform design perspective.
\begin{enumerate}
    \item Linear electromagetic actuators
    \begin{itemize}
        \item solenoid:
        \begin{itemize}
            \item can leverage resonance, large output for small input
            \item force dependent on position within magnetic field
            \item influenced by device orientation relative to gravity
            \item heats up during use
        \end{itemize}
        \item voice coil:
        \begin{itemize}
            \item linear dynamics yields consistent output, relatively easy to model
            \item \textit{C2 tactor}:
            \begin{itemize}
                \item 7.6mm contactor preloaded against the skin
                \item suspension resonates at 250Hz for maximum perceptibility
            \end{itemize}
            \item \textit{Haptuator}:
            \begin{itemize}
                \item moving magnet design
                \item not meant to touch the skin
                \item optimized to render frequencies above 50Hz
            \end{itemize}
        \end{itemize}
    \end{itemize}
    \item Rotary Electromagnetic Actuators (ERM - eccentric rotating mass)
    \begin{itemize}
        \item simple, reliable, rotate continuously with a constant voltage/current applied
        \item off-center mass affixed to output shaft so that its rotation exerts large radial forces on the body of the motor
        \item couples freq and amplitude of the resulting vibration to the motors rotational speed
        \item small voltage yields weaker vibrations
        \item intrinsic spin-up time could cause delay at the start of the cue
        \item internal static friction can prevent motor from rotating when the applied voltage is very small
    \end{itemize}
    \item Nonelectromagnetic Actuators - Piezoelectric effect
    \begin{itemize}
        \item respond to inputs very quickly and can output arbitrary waveforms
        \item typically require input on the order of 100V
        \item high stiffness of skin creates a need for relatively heavy vibrotactile actuator
        \item most don't have power to move the skin without pushing off a cumbersome mechanical ground
    \end{itemize}
    \item EAP (electroactive polymer) actuators
    \begin{itemize}
        \item uses elastomers rather than ceramics
        \item can achieve larger deformations for lower drive voltages
    \end{itemize}
    \item SMA (shape memory allow) actuators
    \begin{itemize}
        \item remembers original shape
        \item mechanical properties altered in response to temp changes
        \item slow response time, large hystoresis, high energy consumption
    \end{itemize}
    \item Pneumatic systems
    \begin{itemize}
        \item compact, light
        \item require high-pressure air source
        \item struggle to output high-frequency signals
    \end{itemize}
    \item Forced impact
    \begin{itemize}
        \item TacHammer - new technology, specs unknown, hard to acquire
    \end{itemize}
\end{enumerate}

\subsubsection{Vibrotactile Constraints}
\begin{enumerate}
    \item Create consistent mechanical coupling between actuator vibrations and users skin
    \item Slight changes to such a system drastically affect users ability to feel and comprehend the rendered signals.
    \item For fixed actuator size/activation level, magnitude of created vibrations is inversely proportional to the mass of the object.
    \item High bandwidth accelerometer can be used to measure vibration output performance. \cite{ignoto2017development}
    \item When the application involves a large object, a wearable device, and/or multiple stimulation sites, the optimal vibrotactile rendering paradigm is to vibrate one or more small zones. 
    \begin{itemize}
    \item For example, in a tactile display application the localization accuracy of 250-Hz vibrotactile stimuli around the waist was 74\% with 12 equidistant tactile actuators (tactors), 92\% with eight tactors, and 97\% with six tactors.\cite{choi2013vibrotactile}	
    \end{itemize}
\end{enumerate}

\section{Design requirements}

The initial requirement set forth by Professor Neely in the Haptic Enviro-Sensing Metronome (HESM) design draft is centered around an analogue wave that could squeeze and release. As the analogue wave approaches its crest it provides insight forecasting the approaching \textit{crusis}, allowing the user to prepare for and rebound from the "click-moment" with rich entrainment. 

This observation is in direct parallel to external vibrotactile metronome research as discussed in \ref{vibrotactileMetronome}. The constraint was such that the pulses should feel continuous and not discrete, even mentioning a pendulum motion as the descriptive feeling to convey. 

As the intention is to encourage entrainment of the human body to external forces, the frequencies
required are quite low, based on the tempi of slow walking to running gaits
(40 bpm/.67 Hz to 180 bpm/3 Hz).

\cite{Neely}
\todo[inline]{add HESM design to bio for citations}

\section{Initial Prototypes}
In order to capture the sensation defined in the design requirements, a series of prototypes were rapidly developed.

\subsection{Solenoid bracelet}
Initial introspection towards capturing the squeeze and release sensation led to the rapid prototyping of a simple solenoid bracelet. 
\subsubsection{Parts List}
\begin{enumerate}
    \item Adafruit Pro Trinket 5V 16MHz
    \item N-channel MOSFET
    \item 1N4004 diode
    \item mini push-pull solenoid
\end{enumerate}
\subsubsection{Assembly}
The design was inspired and assembled per \textit{Adafruit} specification \cite{Solenoid}. 
The base of an N-channel MOSFET was connected through a 1K resistor to a digital i/o pin on the trinket per Figure \ref{SolenoidSchematic}. The collector was connected through the solenoid and diode in parallel to Vcc running at 5V.
\begin{figure}[H]
    \includegraphics[width=\linewidth,height=\textheight,keepaspectratio]{Solenoid_Schematic}
    \caption{Solenoid Schematic}
    \label{SolenoidSchematic}
\end{figure}
\subsubsection{Method}
    As a voltage is applied the slug in the middle of the solenoid is pulled into the center of the coil. The actuation pulls a taught wristband attached to the chasis of the solenoid and as the voltage drops the solenoid retracts releasing tension in the wristband, shown in Figure \ref{SolenoidProto}.
    
    \begin{wrapfigure}[15]{R}{.5\textwidth}
        \centering
        \caption{Solenoid Wristband Prototype}
        \includegraphics[width=.5\linewidth,height=.5\textheight,keepaspectratio]
        {SolenoidProto}
        \label{SolenoidProto}
    \end{wrapfigure}
    
    This was controlled in the \textit{Arduino IDE} through a simple PMW signal with increasing duty cycle which output through the digital I/O. The delay was hard coded proportional to the desired BPM.

\subsubsection{Outcome}

Due to the linear relationship between current draw and pull force, the solenoid required high current and significant voltage thus isolation from the microcontroller was ideal. The necessary rigidity of the band was a cause of discomfort and the lack of positioning options was a detriment to musicians who relied on availability of their hand. Additionally, heat dissipation was at times unsafe and unbearable since the chasis was in direct contact with the skin. Though it captured the tension and release sensation well, there seemed to be a lack of clarity with regards to communication of whether each push pull iteration was a beat length or if a single contraction was the downbeat (i.e. eighth note pulse rather than quarter note). Coupled with the bulky nature of the solenoid chasis, high power requirements and excessive heat dissipation, the solenoid prototype was quickly abandoned.
\subsection{Single vibrotactile}

The subsequent prototype iteration was the first involving a vibrotactile motor. Since the goal was to run everything off of a single board, the voltage constraint was limited to the 5V maximum provided by the \textit{Adafruit Pro Trinket} spec. An ERM motor was chosen for its working voltage range of 2-5V and minimal coin cell form factor (10mm diameter). Like the solenoid, higher applied voltage yield more current draw but stronger vibration. At 5V, a single motor draws 100mA. The specification was 1100 at 5V which roughly translates to 183Hz. Though not quite at the ideal 250Hz range optimal for skin sensitivity, this was deemed close enough.

To realize the spectrum of capability for vibrotactile sensation (beyond pulse width modulation of the signal) a haptic motor controller with a pre-installed library of effects was acquired. 

The goal of this design was to test the ERM sensation on a portable wearable. The MCU was altered from the Pro Trinket to the Flora which ran at 3.3V and had less digital I/O pins, but supported external connectivity and took up less surface area.

\subsubsection{Parts List}
\begin{enumerate}
    \item Vibrating mini motor disc
    \item Adafruit DRV2605L Haptic Motor Controller
    \item Flora Wearable Bluefruit LE Module
    \item Flora Wearable electronic platform
    \item LiPo Battery - 3.7v 1100mAh
\end{enumerate}
\subsubsection{Assembly}
    First, the ERM leads were soldered to the DRV2605 haptic motor controller and connected via I2C protocol to the complimentary pins on the Flora (SCL,SDA).
    To experiment with wirelessly triggering the vibrotactile, the Bluefruit low energy module was added and the send and receive (Tx/Rx) pins were connected as referenced in Figure \ref{Proto2}. Last, the battery was connected via the built-in terminal clip.


    \begin{figure}[H]
        \includegraphics[width=\linewidth,height=\textheight,keepaspectratio]{Proto2_bb}
        \caption{Prototype 2 - Single vibrotactile, wireless connectivity}
        \label{Proto2}
    \end{figure}

\subsubsection{Method}
    Once the hardware was setup, the haptic library was iterated through. The optimal sensation chosen was a queue of two chained effects according to the \underline{DRV2605 datasheet}:
    \begin{itemize}
        \item 83 - Transition Ramp Up Long Smooth 2 - 0 to 100\%
        \item 71 - Transition Ramp Down Long Smooth 2 - 100\% to 0
    \end{itemize}

    Within the Arduino IDE the Bluefruit library and dependencies were imported configured via UART. The connectivity was validated via the publicly available Bluefruit LE application on an external Android device.

    The code was written such upon setup and BT initialization, the main loop was constantly listening for updates over UART. The app would send integer values representing the desired BPM to the connected haptic wearable. Once received the new bpm value was parsed into a period value in milliseconds via equation \ref{eq:period}

    \begin{equation}
        \label{eq:period}
        period = \frac{60,000}{bpm}
    \end{equation}

    Since the highest operational bpm defined would be 180, the shortest period would be an interval of 333.33 ms. This value divided in half gave the maximum allowed ramp up time for the motor. The new period value was fed into a state machine which set the on and off state of the motor based on a timer from half the calculated period.

    \subsubsection{Outcome}
    The singular ERM was a valid starting point but the downfall of this prototype was it's inability to fully command the wearers attention.

    The next iteration needed to operate at higher voltage to get a stronger vibration. 

\section{Vibrotactile Haptic Array}

\subsection{Hardware}

\begin{figure}[H]
    \includegraphics[width=\linewidth,height=\textheight,keepaspectratio]
    {FinalProto_bb}
    \caption{Final prototype wiring mockup}
    \label{FinalProto}
\end{figure}

\subsection{Software}

\section{Design Challenges}

Motor noise

Managing power dips



Debounce for tap tempo

\section{Optimization}

\subsection{Future Implementation}

Bluetooth/Wireless

Custom PCB

Experiment with other vibrotactiles such as tachammer and LNA's
