\chapter{Method}
This chapter outlines each test case, describes the motivation behind the test plan, and delves into both the hardware and software design required to implement the methodology. Furthermore, it defines and calculates a round-trip latency in the system to find an approximation of relative accuracy. The hope is to establish a level of confidence in the precise time dependent information.

The overall test principle was derived from traditional sensorimotor synchronization tasks in which a user is asked to tap to a corresponding stimulus. The asynchrony was tracked and plotted along with the PCR and any missed taps. Since the haptic domain is of primary focus, the auditory modality functions primarily as a benchmark or baseline foundation. The work presented in \ref{visualMet} covers the idea of the interstitial beat occupying the visual domain and as such will not be re-evaluated here.

Each test case is defined and presented in \ref{testPlan}. The overall software development is detailed in \ref{development}. The hardware setup and code re-purposing from TapArduino is discussed in \ref{tap_arduino} and latency calculated in \ref{latencyCalc}.

\section{Test Plan} \label{testPlan}
Testing was divided into two major sections, \textbf{Steady} and \textbf{Dynamic}, implying either an \textit{isochronous} or a \textit{non-isochronous} pulse respectively. While structurally identical, the dynamic tests however focussed on rubato within a range starting at the predefined BPM and rising or falling within a specified window. The chosen tempi parallels slow walking to running gaits spanning a range of 45-180 beats per minute.

Each section had three subsections centered around either an audible metronome tone (\textbf{A1, A3}), musical note (\textbf{A2, A4}), and lastly the haptic modality (\textbf{H1, H2}). Subsections were further broken down into \textbf{a} and \textbf{b}, denoting either \textit{discrete} or \textit{interstitial}/\textit{continuous} mode of operation. A breakdown of the test plan is shown in Figure \ref{fig:TestPlan}.

As discussed in Chapter 3, the haptic was designed with two operating modes in mind, discrete and continuous. These modes were programmatically controlled to match the desired test cases, extensively explained in section \ref{development}.
\begin{table}[]
    \centering
    \resizebox{\textwidth}{!}{%
    \begin{tabular}{cclllcclll}
    \hline
    \multicolumn{10}{c}{Steady} \\ \hline
    \multicolumn{3}{c}{Discrete} & BPM & Runtime (sec) & \multicolumn{3}{c}{Interstitial} & BPM & Runtime (sec) \\
    \multirow{4}{*}{A1a} & \multirow{4}{*}{click} & i. & 45 & 20 & \multirow{4}{*}{A1b} & \multirow{4}{*}{legato chime (swing click)} & i. & 45 & 30 \\
     &  & ii. & 90 & 20 &  &  & ii. & 90 & 16 \\
     &  & iii. & 135 & 20 &  &  & iii. & 135 & 11 \\
     &  & iv. & 180 & 20 &  &  & iv. & 180 & 8 \\
    \multirow{4}{*}{A2a} & \multirow{4}{*}{staccato music (melody)} & i. & 45 & 32 & \multirow{4}{*}{A2b} & \multirow{4}{*}{legato music (melody)} & i. & 45 & 32 \\
     &  & ii. & 90 & 16 &  &  & ii. & 90 & 16 \\
     &  & iii. & 135 & 11 &  &  & iii. & 135 & 11 \\
     &  & iv. & 180 & 8 &  &  & iv. & 180 & 8 \\
    \multirow{4}{*}{H1a} & \multirow{4}{*}{poke / all on (instantaneous)} & i. & 45 & 15 & \multirow{4}{*}{H1b} & \multirow{4}{*}{oscillate down and back up} & i. & 45 & 15 \\
     &  & ii. & 90 & 15 &  &  & ii. & 90 & 15 \\
     &  & iii. & 135 & 15 &  &  & iii. & 135 & 15 \\
     &  & iv. & 180 & 15 &  &  & iv. & 180 & 15 \\ \hline
    \multicolumn{10}{c}{Dynamic} \\ \hline
    \multicolumn{3}{c}{Discrete} & BPM & Runtime (sec) & \multicolumn{3}{c}{Interstitial} & BPM & Runtime (sec) \\
    \multirow{4}{*}{A3a} & \multirow{4}{*}{click} & i. & 45 +/- 15 & 20 & \multirow{4}{*}{A3b} & \multirow{4}{*}{legato chime (swing click)} & i. & 45 +/- 15 & 20 \\
     &  & ii. & 90 +/- 15 & 10 &  &  & ii. & 90 +/- 15 & 10 \\
     &  & iii. & 135 +/- 15 & 10 &  &  & iii. & 135 +/- 15 & 10 \\
     &  & iv. & 180 +/- 15 & 10 &  &  & iv. & 180 +/- 15 & 10 \\
    \multirow{4}{*}{A4a} & \multirow{4}{*}{staccato music (melody)} & i. & 45 +/- 15 & 30 & \multirow{4}{*}{A4b} & \multirow{4}{*}{legato music (melody)} & i. & 45 +/- 15 & 30 \\
     &  & ii. & 90 +/- 15 & 15 &  &  & ii. & 90 +/- 15 & 15 \\
     &  & iii. & 135 +/- 15 & 10 &  &  & iii. & 135 +/- 15 & 10 \\
     &  & iv. & 180 +/- 15 & 10 &  &  & iv. & 180 +/- 15 & 10 \\
    \multirow{4}{*}{H2a} & \multirow{4}{*}{poke / all on (instantaneous)} & i. & 45 +/- 10 & 15 & \multirow{4}{*}{H2b} & \multirow{4}{*}{oscillate down and back up} & i. & 45 +/- 10 & 15 \\
     &  & ii. & 90 +/- 5 & 15 &  &  & ii. & 90 +/- 5 & 15 \\
     &  & iii. & 135 +/- 3 & 15 &  &  & iii. & 135 +/- 3 & 15 \\
     &  & iv. & 180 +/- 1 & 15 &  &  & iv. & 180 +/- 1 & 15
    \end{tabular}%
    }
    \caption{Test Plan}
    \label{fig:TestPlan}
\end{table}

\subsection{Audio File Generation}
All tracks were rendered using the digital audio workstation (DAW) \textit{Logic Pro X} as \textit{.wav} files at a 44.1kHz sample rate with 16 bit resolution.
\subsubsection{Metronomic click and legato chime}
\begin{wrapfigure}{r}{5cm}
    \centering
    \includegraphics[width=0.25\textwidth]{klopfgeist}
    \caption{Default metronome}
    \label{fig:klopfgeist}
\end{wrapfigure}
\textbf{A1a} and \textbf{A3a} required a standard metronomic pulse. This was accomplished using the default Klopfgeist (metronome) plugin from Logic Pro X. No additional tuning was modified and the tonality was left at 0.83 of unity as shown in \ref{fig:klopfgeist}. 

\textbf{A1b} and \textbf{A3b} however required a swing or legato type of chime in order to convey filling the interstitial space. To capture this effect the Klopfgeist tonality was increased to unity and tuned -27 semitones lower which served to both soften diminish the discrete click, provided an elongated or continuous audible sensation. To give the impression of a sound that was ramping up in amplitude and decaying after the peak, a tremolo effect which mimics a sawtooth wave was added to the signal chain as seen in \ref{fig:modClick}. Last, a multi-band EQ was placed at the end of the signal chain with a bandpass filter from 95Hz-750Hz to remove unwanted frequency presence and a 3.5dB high-Q peak at 220Hz to emphasize the tonality.

\begin{figure}[H]
    \centering
    \caption{Modified click parameters for interstitial tests.}
        \subfloat[Modified metronome]{\includegraphics[width=0.25\columnwidth]{Klopfgeist_Modified}}
        \qquad
        \subfloat[Superimposed tremolo]{\includegraphics[width=0.4\columnwidth]{Tremolo}}
        \qquad
        \subfloat[Equalized tone]{\includegraphics[width=\textwidth,height=0.25\textheight]{Modified_Click_EQ}}
    \label{fig:modClick}
\end{figure}

The resultant waveform encapsulated the occupation of the interstitial space. A comparison of this waveform in contrast to it's discrete counterpart is shown in \ref{fig:click_comparison}. Note the envelope of signal (b) follows a natural build up and decay.

\begin{figure}[H]
    \centering
    \caption{Metronomic waveform comparison}
        \subfloat[A3a1: discrete audible click]{\includegraphics[width=0.5\columnwidth]{Click_waveform}}
        \subfloat[A3b1: interstitial tone]{\includegraphics[width=0.5\columnwidth]{SwingClick_waveform}}
    \label{fig:click_comparison}
\end{figure}

\subsubsection{Stacatto and legato melody}
As a specific musical listening task, test cases \textbf{A2a}, \textbf{A4a} and \textbf{A2b}, \textbf{A4b} involve synchronization to a simple melodic sequence of notes. The music chosen was the nursery rhyme \textit{Pat-A-Cake}. The initial mockup was drafted in Sibelius and exported to Logic Pro X for bpm adjustment.

Each quarter note represents a beat and therefore a tap onset. In order to emphasize a discrete event for test cases \textbf{A2a} and \textbf{A4a}, notes were input as stacatto, shown below in Figure \ref{fig:patacakea2a}.

\begin{figure}[H]
    \centering
    \includegraphics[width=\textwidth]{Pat-a-Cake_a2a}
    \label{fig:patacakea2a}
\end{figure}

The interstitial counterparts (\textbf{A2b}, \textbf{A4b}) similarly underwent crescendo and decrescendo after every note onset with forte accents surrounded by mezzopiano to give the impression of amplitude build up and decay, shown below: \ref{fig:pat-a-cake_a2b} 

\begin{figure}[H]
    \centering
    \includegraphics[width=\textwidth]{Pat-a-Cake_A2b}
    \label{fig:pat-a-cake_a2b}
\end{figure}

Note below in Figure \ref{fig:music_comparison} the gradual, nearly exponential decay displayed in the interstitial tone as a result of the legato input.

\begin{figure}[H]
    \centering
    \caption{Musical waveform comparison}
        \subfloat[A2a1: stacatto melody]{\includegraphics[width=\columnwidth]{A2a1_waveform}}
        \qquad
        \subfloat[A2b1: legato melody]{\includegraphics[width=\columnwidth]{A2b1_waveform}}
    \label{fig:music_comparison}
\end{figure}

\subsubsection{Dynamic tempi manipulation - audio}
Dynamic manipulation of tempo was accomplished in \textit{Logic Pro X} through automation of the tempo parameter over the time period of the desired waveform. Each test case started on one of the pre-defined BPM's (45, 90, 135, 180) but traversed either sinusoidally or triangularly through segmented time blocks as peaks and troughs ranging plus or minus 15 bpm; shown in \ref{fig:dynamic_audio}.

\begin{figure}[H]
    \centering
    \caption{Dynamic audio tempo automation patterns}
        \subfloat[45 +/- 15]{\includegraphics[width=.5\columnwidth]{dynamic_45}}
        \subfloat[90 +/- 15]{\includegraphics[width=.5\columnwidth]{dynamic_90}}
        \qquad
        \subfloat[135 +/- 15]{\includegraphics[width=.5\columnwidth]{dynamic_135}}
        \subfloat[180 +/- 15]{\includegraphics[width=.5\columnwidth]{dynamic_180}}
    \label{fig:dynamic_audio}
\end{figure}

\section{Test Suite}
High precision data acquisition and the minimization of delay were the central foci of the test suite design. Due to the extensive amount of publicly available libraries, multithreading capability, pandas dataframe structure, and plot integration via matplotlib, \textit{Python} was chosen as the development environment. Complementary to the software platform was the implementation of a tap onset detection mechanism via force sensitive resistor (FSR) and the \textit{Arduino Uno}. 

\subsection{Software Development} \label{development}
\todo[inline]{fill out section}
            Discuss code breakdown
            
            GUI
            
            Haptic onset detection
            
            Tap onset detection
            
            Multithreading
            
            Audio onset detection
            
            Plotting

\subsection{Tap Onset Hardware}    \label{tap_arduino}
\todo[inline]{fill out section}

\subsection{Latency Evaluation} \label{latencyCalc}
Overall strategy to minimize delay maximize accuracy/precision
\todo[inline]{fill out section}
Sources of potential error:

    *FTDI/USB -> Pro Trinket (16MHz) --> Laptop

    *USB -> Arduino -> Laptop

    *Serial write to haptic to motor spin up

    *FSR analog read/mentioned debounce

    *python thread time

    Evaluation with scope:

\subsection{Setup} \label{testSetup}
To initialize setup, the user is seated and given a pair of closed-back headphones. The FSR is situated to their preference, either dominant or non-dominant hand, and secured into place. Unlike a keyboard or button the FSR gives no feedback or rebound; psychologically ensuring a confident tap on each onset while providing no tactile response. This approach seeks to avoid intrinsic lag from its independance of mechanical components. The delay limit is defined by the threshold applied in the software to avoid debounce, as discussed in Section \ref{development}

The user will input their name, read the instructions, agree to the conditions of the test suite, and commence with the test. After every iteration of the test, a plot of the Tap Onset, True Onset, and Sanitized Onset is presented to the user for the purposes of feedback and affirmation of correct tapping. The order every user encounters will differ as the tests are scrambled. Upon completion, the users are asked to fill out a survey and the results are displayed.

\begin{figure}[H]
    \centering
    \includegraphics[width=\columnwidth]{TestSuiteFlowDiagram}
    \caption{Test Suite Flow Chart}
    \label{fig:TestSuiteFlowDiagram}
\end{figure}

\subsubsection{Example Output}
\todo[inline]{fill out section}

\section{Expectations}
\todo[inline]{fill out section}